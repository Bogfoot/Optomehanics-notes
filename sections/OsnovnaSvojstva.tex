Gledamo prvo klasični odziv jednostavne Fabry-Perot (FP) rezonirajuće komore. FP rezonantna šupljina sastoji se od dva visoko reflektivna
zrcala udaljena L jedno od drugog, i niz rezonancija dane s kružnom frekvencijom $\omega_{cav,m} \approx \frac{m\pi c}{L}$. Ovdje je \textit{m} cijeli broj koji označava
vibracijski mod. Razmak između dva logitudinalna rezonantna moda je dan s
\begin{equation}
	\delta\omega_{FSR} = \pi \frac{c}{L},
\end{equation}
%Nacrtaj kako izgleda neki spektar
gdje $\delta \omega_{FSR}$ označava slobodni spektralni raspon, odnosno raspon frekvencija kojim naš rezonator ne vibrira. Konačna transparentnost zrcala i
interna absorpcija ili raspršenje van rezonantne šupljine dovode do konačnog fotonske stope istjecaja $\kappa$. 
Korisna je znati i optičku kvalitetu (eng. \textit{finess}) $\mathfrak{F}$ naše šupljine koja obilježava srednji broj refleksija fotona prije nego izađe iz šupljine. Dana je s 
\begin{equation}
	\mathfrak{F} = \frac{\delta\omega_{FSR}}{\kappa}.
\end{equation}
Optička kvaliteta je bitna za određivanje snage unutar komore.
\begin{exampleblock}{Primjer}

	
\end{exampleblock}
