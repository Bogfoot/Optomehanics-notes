\subsection{Ulazno-Izlazni formalizam optičke šupljine}
Kvantno mehanički opis rezonantne šupljine vezane za vanjsko elektromagnetsko zračenje može se dati ili koristeći tzv. \textit{Master} jednadžbe (ako nas
zanima samo unutarnja dinamika) ili, ako želimo saznati EM polje emitiranog (reflektiranog) zračenja u šupljini, pomoću ulazno-izlaznog formalizma. 
Taj formalizam nam dopušta modeliranje kvantnih fluktuacija iz bilo kojeg terminala vezanja (npr. ulazno zrcalo) šupljine. Također se uzima u obzir koherentno
zračenje kojim "guramo" sustav\footnote{Baciti oko na \cite{gardiner_zoller_2004} i \cite{clerk_devoret_girvin_marquardt_schoelkopf_2010}.}.\\
Korištenjem Heisenbergovih jednadžbi gibanja opisujemo vremensku evoluciju amplitude $\hat{a}$ unutar šupljine. Amplituda se guši s $\frac{\kappa}{2}$. Istovremeno
se fluktuacije konstantno obnavljaju kroz ulaze u šupljinu putem kvantnog šuma. Razlikujemo između dva kanala ulaznog vezanja ($\kappa_{ex}$) i ostalih gubitaka ($\kappa_0$).
\begin{equation}
\dot{\hat{a}}=-\frac{\kappa}{2} \hat{a}+i \Delta \hat{a}+\sqrt{\kappa_{\mathrm{ex}}} \hat{a}_{\mathrm{in}}+\sqrt{\kappa_0} \hat{f}_{\mathrm{in}}
\label{eq:1}
\end{equation}
U klasičnom slučaju $\hat{a}$ bi zamjenili svojstvom normalizacije kompleksne amplitude električnog polja moda šupljine koji nas zanima. 
Prelazak na kasični slučaj možemo dobiti usrednjavanjem tako da $\hat{a} \mapsto \left<\hat{a}\right>$. Odabrali smo rotirajući referentni sustav s frekvencijom $\omega_L$ odnosno,
$\hat{a}^{ishodište} = e^{-i\omega_L t}\hat{a}^{\textbf{tu}}$, gdje uvodimo tzv. \textit{laser detuning} $\Delta = \omega_L - \omega_{cav}$ u odnosnu na mod šupljine. 
Ulazno polje $\hat{a}_{in}(t)$ gledamo kao stohastičko\footnote{\href{https://en.wikipedia.org/wiki/Random_field}{Nasumično/stohastičko polje}} kvantno polje. U najjednostavnijem slučaju predstavlja vakuumske fluktuacije 
električnog polja vezanog za šupljinu u trenutku t zajedno sa koherentnim laserskim pogonom. Isti formalizam se može koristiti za opis \textit{squeezanih} stanja, odnosno, bilo kakvih kompleksnijih stanja polja. 
Polje je normalizirano na način da 
\begin{equation}
	P = \hbar \omega_L \braket{\hat{a}_{in}^{\dagger}\hat{a}_{in}},
	\label{eq:2}
\end{equation}
gdje je P snaga ulaznog zračenja u šupljinu a $\braket{\hat{a}_{in}^{\dagger}\hat{a}_{in}}$ je stopa ulaznih fotona u šupljini. Isti opis vrijedi za $\hat{f}_{in}$.
Polje reflektirano od FP rezonatora (ili vezano nazad u vezajući valovod) dano je s 
\begin{equation}
	\hat{a}_{out} = \hat{a}_{in} - \sqrt{\kappa_{ex}}\hat{a}.
	\label{eq:3}
\end{equation}
Jednadžba \ref{eq:3} opisuje i jednosmjerne valovod-rezonator sustave poput \textit{whispering-gallery-mode} rezonatora\footnote{\href{https://journals.aps.org/prl/abstract/10.1103/PhysRevLett.85.74}{Cai, Painter, Vahala, 2000}}.
Prvo ćemo gledati klasične usrednjene vrijednosti za jednostranu šupljinu. Usrednjimo \ref{eq:1} i \ref{eq:2}. Iz jednadžbu \ref{eq:1} prvo možemo dobiti amplitudu za slučaj kad 
nemamo promjena u sistemu u prisutnosti monokromatskog pogonskog zračenaj (engl. \textit{steady state}) čija je amplituda dana sa $\braket{\hat{a}_{in}}$. Recimo da je $\braket{\hat{f}_{in}} = 0$ i dobivamo
\begin{equation}
	\braket{\hat{a}} = \frac{\sqrt{\kappa_{ex}^{(2)}}\braket{\hat{a_{in}}}}{\frac{\kappa}{2} - i\Delta}.
	\label{eq:4}
\end{equation}
Izraz koji povezuje ulazno polje za polje u šupljini nazivamo optičkom suseptibilnošću,
\begin{equation}
	\chi_{opt}(\mathbf{\omega}) \equiv \frac{1}{-i(\omega+\Delta)+\kappa/2}.
	\label{eq:5}
\end{equation}
Ovdje je $\omega$ Fourierova frekvencija fluktuacija ulaznog polja oko laserske frekvencije $\omega_L$. Ovaj jednostavni Lorenzijanski odziv je aproksimacija i ignorira sve druge rezonancije šupljine. 
Dok god je stopa raspada $\kappa$ mnogo manja od udaljenosti između rezonancija ($\omega_{FSR}$) ova aproksimacija je adekvatna, što bi značilo da je optički faktor kvalitete $\mathcal{Q}_{opt}$ visok. 
Steady-state naseljenost šupljine 
\begin{equation}
	n_{cav} = \braket{\hat{a}^{\dagger}\hat{a}},
	\label{eq:6}
\end{equation}
odnosno srednji broj fotona u šupljini dan je s 
\begin{equation}
	n_{cav} = \braket{\hat{a}^{\dagger}\hat{a}} = \frac{\kappa_{ex}P}{\hbar\omega_L(\Delta^2+(\kappa/2)^2)},
	\label{eq:7}
\end{equation}
gdje je P ulazna snaga u šupljinu definirana s \ref{eq:2}. Ubacimo li jednadžbu \ref{eq:4} u \ref{eq:3} dobivamo reflektivnost ili transmisiju amplitude. Označimo reflektivnost amplitude s $\mathcal{R}$,
\begin{equation}
	\mathcal{R} = \frac{\braket{\hat{a}_{out}}}{\braket{\hat{a}_in}} = \frac{(\kappa_0 -\kappa_{ex})/2 -i\Delta}{(\kappa_0 +\kappa_{ex})/2 -i\Delta}.
	\label{eq:8}
\end{equation}
Kvadrat ove reflektivnosti $\mathcal{R}^2$ daje vjerojatnost refleksije o šupljinu ili transmisiju u slučaju sustava jednosmjerne valovod-rezonator šupljine.
Iz ovog izraza možemo raspoznati nekoliko režima. 
\begin{enumerate}
	\item $\kappa_{ex} \approx \kappa >>\kappa_0$ - $\kappa_{ex}$ dominira gubitke gubitke šupljine, šupljinu nazivamo "prevezanom". U tom slučaju je $|\mathcal{R}|^2 \approx 1$ 
i fotoni pumpe izađu iz šupljine bez da apsorpcije ili izgubljeni preko drugog zrcala (limit kvantne detekcije).
	\item  $\kappa_0 = \kappa_{ex}$ - Kritično vezanje. U ovom slučaju $\mathcal{R} (\Delta = 0) = 0$ na rezonanciji. Dolazi do kompletne disipacije snage u rezonatoru, ili je potpuno transmitirano kroz drugo zrcalo.
	\item $\kappa_{ex} << \kappa_0$ - "Podvezano" stanje. Dominiraju intrinzični gubici šupljine. Često zanemarivo stanje jer vodi do efektivnog gubitka informacija o sustavu.
\end{enumerate}

\begin{Bilješka}
	Pod "\textit{steady-state}" misli se na to da je sustav u nekoj vrsti ekvilibrija, zamislite laminarni tok vode kroz rupu.
\end{Bilješka}
